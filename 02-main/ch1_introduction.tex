% ------------------------------------------------------------------------------
% Introduce the topic - What characterizes the topic?
% Introduce the goal - What do you want to achieve with your thesis?
% Make the reader curious - What motivates the reader to read on?
% Describe the relevance - Why is this bachelor thesis scientifically relevant?
%
% The introduction should have the following content:
% - Initial situation & presentation of the topic - You introduce the topic with an exciting 'bait'. You provide initial information on the topic and the object of research and explain the current state of research.
% - Relevance of the topic & motivation - You justify the relevance of your topic (scientifically) and place it in the context of your field. In addition, it is often required that you disclose your personal motivation.
% - Problem description and thematic delimitation - By means of a specific research question (or hypothesis) you present your explicit research interest. If necessary, explain technical terms.
% - Objectives - Your introduction should clearly state what the goal of your paper is and what outcome you hope to achieve upon completion of the bachelor thesis.
% - Method You explain the approach and justify the choice of method.
% - Structure of the Bachelor's thesis - Finally, you give the reader a general overview of your Bachelor's thesis by explaining the structure, showing the red thread and how the research question is answered.
% ------------------------------------------------------------------------------

\opt{never}{\addbibresource{03-tail/bibliography.bib}} % to make citation found in most IDE

\chapter{Introduction}
\label{chap:introduction}

Term (glossaries): \gls{scrum}

Acronym (glossaries): \acrfull{ar}

Citation (biblatex): \cite{stateoftheArt}

% -----------------------------------------------------------------------------
\section{Context / Problem}

% -- Your text goes here --
\lipsum[1]

% -----------------------------------------------------------------------------
\section{Objectives}

% -- Your text goes here --
\lipsum[2]

% -----------------------------------------------------------------------------
\section{Structure of this report}

% -- Your text goes here --
\lipsum[3]
